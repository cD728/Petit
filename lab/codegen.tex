% the first exercise should be a manual LLVM IR implementation: an interesting one would be manual gcd implementation that then is the test case they need their compiler to compile at the end of the class

Restriction: We shall consider only type integer for all of the following exercises. We ignore the existence of doubles altogether.
\begin{enumerate}
\item Implement code generation for multiplication.
\item Implement code generation for the other three arithmetic operations.
\item Implement code generation for using double variables:
\begin{enumerate}
\item Allocate a stack space for each variable.
\item Modify the code so that assignments to double variables are allowed.
\item Implement the case of identifiers used in expressions.
\end{enumerate}
\item According to the language semantics, it is possible to assign integer values to double variables and \textit{vice versa}. In both cases it is necessary to make the appropriate conversion. Natural constants must be converted to double before assigning their value to a variable of double type; and Decimal constants must be converted to integer before assigning their value to an integer variable. Implement these two cases.
\end{enumerate}

% show a simple loop in llvm?
% show how to link a Petit program to read() and write() functions written in C

% code for If: we could use llvm ir 'select' but that's not general (it would specifically work here)

% read and write functions are only added here for code generation; not before; with the note to add to the symbol table and that we need to 'define'